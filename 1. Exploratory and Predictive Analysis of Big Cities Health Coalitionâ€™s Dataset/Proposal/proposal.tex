\documentclass[11pt]{article}


\usepackage[margin=0.5in]{geometry}
\usepackage{authblk}
\usepackage{graphicx}
\usepackage{hyperref}
\usepackage{float}

\title{\textbf{DS 5110 Project Proposal} \\ Exploratory and Predictive Analysis of Big Cities Health Coalition’s Dataset}
\author{Kunjan Khatri, Nikson Panigrahi, Prakhar Patidar, Rishabh Shanbhag }
\date{}
\affil{Northeastern University, Boston, Massachusetts}

\begin{document}
  \maketitle
  \pagenumbering{gobble}
  \section{Summary}
  The data set is based on the Big Cities Health Inventory (BCHI) which is an open data platform managed by Big Cities Health Coalition (BCHC). It contains a collection of data representing a snapshot from 30 large cities in the United States of America.  The objective of this project is to gain insights from the health dataset, find out which diseases are more prevalent and major cause of mortality in urban environment, raise awareness about certain diseases which are not familiar to general public, try to predict which disease a person is likely to have given a set of variables like city, age, sex.
  \\
  The dataset is available in .csv format containing 15 columns/variables which includes indicator category, indicator, age, year, sex, race/ethnicity, value, place, etc. The categories of diseases are merged in observations and hence will require tidying up the data a little bit to create variables of interest. Apart from this a certain number of data preprocessing steps are required to clean up the data and produce insights from it. Consequently, it is planned to produce various insights from the data in the form of visualizations and find correlations and other patterns in the data, to help in modelling and produce interesting results. 

  \section{Proposed Plan of Research}
  The plan of approach is to utilize this data to produce non-trivial inferences by waging the indicator data to different causal variables. The process starts from cleaning the data to increase the usability and understandability by cleaning out the empty fields, manipulation of the records for variables such as Methods, Notes etc. to replace the sentences with some meaningful strings which would provide better setting for analysis. The time series data (years) would be used to analyze the distribution of diseases based on the confidence levels over the years using visualization techniques and produce inferences.  Additionally, we plan to use visualization techniques to express the mortality rates for different indicator categories distributed over the cities. The dataset mostly contains categorical variables and a few continuous variables so the plan is to create classification models from the data, which would be used to predict diseases a person is likely to have based on various given factors. 

  \section{Preliminary Results}
  Successfully loaded dataset into R and began performing exploratory analysis to understand the overall scope of dataset.
  Performed tidying on data set, cleaned the records with empty demographic values. Started with string extraction on variables to extract essential information and reject the rest to increase understandability of data. From (Figure 1) we notice an increase until the year 2013 and after that a decline is observed in overall count of recorded disease data over the years can be seen. Years 2017 and 2018 do not have enough data according to the dataset documentation and hence cannot be used.

  \newpage
  \begin{thebibliography}{9}
    \bibitem[1]{} \emph{Big Cities Health Coalition (BCHC)} 2010-2016,
    \url{https://www.bigcitieshealth.org/}. 
    \bibitem[2]{} \emph{Centers for Disease Control and Prevention},
    \url{https://www.cdc.gov/brfss/index.html}
  \end{thebibliography}

  \section*{Appendix}
  
  List of Figures:
  
  \begin{figure}[H]
    \centering
    \includegraphics[width=5in]{"year_count"}
    \caption{Trend of Health Problems over the Years}
  \end{figure}

\end{document}